\documentclass[letterpaper,12pt]{article}
\usepackage[utf8]{inputenc}
\usepackage[spanish]{babel} %Idioma
\input{packetes}


\newcommand{\atom}{\textit{atom}(\phi)}
\newcommand{\atomg}{\textit{con}(\gamma)}
\newcommand{\con}{\textit{con}(\phi)}
\newcommand{\icd}{\textit{icd}(\phi)}

%------------------ S T A R -------------------------------------------------------------------|
\begin{document} 
%------------------ TITULO
                \titulo
%------------------ TITULO
\Large{

\section*{Definiciones}
\begin{itemize}
    \item[$\bullet$] \textbf{Alfabeto:} Conjunto finito de caracteres o símbolos denotado por $\Sigma$
    \begin{itemize}
        \item $\Sigma_1 = \{ 0,1 \}$
        \item $\Sigma_2 = \{ a, b, c, ..., z \}$
    \end{itemize}
    
    \item[$\bullet$] \textbf{Cadenas:} secuencia finita de los caracteres definidos en $\Sigma$ 
    \begin{itemize}
        \item $\Sigma_1 = 10101 $
        \item $\Sigma_2 = hola $
    \end{itemize}
    
    \item[$\bullet$] \textbf{Longitud de cadena:} Es el número de elementos que hay en la cadena. se denote con: $|x|$
    \begin{itemize}
        \item $ |10101| = 5 $
        \item $ |hola| = 4 $
    \end{itemize}
    
    \item[$\bullet$] \textbf{Cadena vacía:} Es es tiene longitud $0$, la denotamos con $\varepsilon$, con $|\varepsilon| = 0$ 
\end{itemize}

\section*{Operaciones en cadenas}    

\begin{itemize}

    \item \textbf{Concatenación}
    \begin{itemize}
        \item Se tiene las cadenas $x = 101$ y $y = 111$, la cadena $xy = 101111$ 
    \end{itemize}
    
    \item \textbf{Concatenación de la cadena vacía}
    \begin{itemize}
        \item Sea la cadena $x = 101$ y la cadena vacía $\varepsilon$, la cadena $x\varepsilon = 101$
    \end{itemize}
    
    
    \item\textbf{Elevar una cadena:}  Para la cadena $a^{n}$ será la cadena de $a's$ de longitud $n$, $a^{5} = aaaaa $. Y podemos definir:
    \begin{equation*}
        a^{0} =  \varepsilon
    \end{equation*}
    \begin{equation*}
        a^{n+1} = a^{n}a
    \end{equation*}
    
\end{itemize}

    
\subsection*{Operaciones en conjuntos}        
\begin{itemize}
    \item \textbf{Cerradura de Klenne:}  Se denota para el conjunto del alfabeto como $A^{*}$ y representa la unión de todas las potencias de $A$. Es decir, una concatenación de todas las potencias $(A^{1} \cup A^{2} \cup A^{3}...)$
    
    \item \textbf{Cerradura de Klenne para el conjunto vacío}
    Da como resultado al conjunto con la cadena vaciía 
    \begin{equation*}
        (\emptyset)^{*}  = \{ \varepsilon \}
    \end{equation*}
    
\end{itemize}

\section*{Lenguajes}
\subsection*{Que son los lenguajes}

Al conjunto de cadenas tomadas de un alfabeto $\Sigma^{*}$ se le llama lenguaje. Es decir, podemos tener un alfabeto tal que $\Sigma = \{ a, b, c ...z \}$ (las letras del abecedario), y sabemos que el lenguaje del español esta contenido aquí pero también el lenguaje del inglés y ambos derivan de $\Sigma^{*}$. De modo que si $\Sigma$ es un alfabeto y $L$ es un lenguaje: $L \subseteq \Sigma^{*}$, entonces $L$ es un lenguaje de $\Sigma$

\subsection*{Puntos importantes}

\begin{itemize}

    \item[$\bullet$] $\Sigma^{*}$ Es el lenguaje para cualquier alfabeto $\Sigma$
    
    \item[$\bullet$] El lenguaje vacío, es el lenguaje de cualquier alfabeto
    
    \item[$\bullet$] El lenguaje que consta sólo de la cadena vacía $(\{  \varepsilon \})$, tambien es un lenguaje de cualquier alfabeto. 
    
    \item[$\bullet$] $\emptyset \neq \{\varepsilon\}$. A la izquierda no contiene ninguna cadena. A la derecha contiene sólo una cadena.
    
\end{itemize}

} \end{document}
%----------------------E  N  D-----------------------------------------------------------------|
